\chapter{Considera��es Finais}
Este trabalho trouxe uma apresenta��o da metodologia usada para o restabelecimento autom�tico de energia el�trica considerando o uso de equipamentos telecomandados. Apesar de n�o constar nesse relat�rio, j� existe um sistema computacional que implementa a metodologia apresentada aqui e os resultados est�o satisfat�rios e dentro do esperado.

O desenvolvimento dessa metodologia para restabelecimento autom�tico, traz vantagens em rela��o a outras metodologias \cite{Gris2010} por ser aplic�vel em diferentes topologias de rede sem a necessidade de configura��o das manobras que podem ser executadas. Isso se d� pela forma que essa metodologia utiliza as informa��es em campo (medi��es, controle) com intelig�ncia computacional (simula��es, defini��es das manobras) para obter o conjunto de manobras a serem executadas para restabelecimento m�ximo do fornecimento de energia el�trica para os consumidores.

Poss�veis trabalhos futuros: desconsiderar a restri��o de radialidade e permitir a transfer�ncia para mais de uma alimentador.
Uso de um outro m�todo de defini��o das manobras a serem executadas (fun��o multi objetivo, usando AHP ou outro m�todo), uso de sistemas evolutivos (AG por exemplo) para escolha de quais manobras podem ser executadas.

Uso de Redes neurais para armazenar fatos do passado e se a mesma situa��o ocorrer n�o precisar recalcular/esperar operador aceitar viola��es de restri��o.



